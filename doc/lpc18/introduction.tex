 \section{Introduction}\label{sec:introduction}

(This section is adapted from~\cite{budiu-osr17}.)  One of the most
 active areas in computer networking is Software Defined Networking
 (SDN)~\cite{rfc7426}.  SDN separates the two core functions of a
 network element (e.g., router): the control-plane and the data-plane.
 Traditionally both these functions were implemented on the same
 device; SDN decouples them, and allows multiple control-plane
 implementations for managing each data-plane.  The Open Flow
 protocol~\cite{mckeown-ccr08} is a standard SDN tool.  However, SDN
 still assumes that the behavior of the network \emph{data-plane} is
 fixed.  This is a significant impediment to innovation; for example,
 the deployment of the VXLAN protocol~\cite{rfc7348} took 4 years
 between the initial proposal and its commercial availability in
 high-speed devices.

\cite{bosshart-ccr14} proposed the P4 language: Programming
Protocol-independent Packet Processors, which is designed to make the
behavior of \emph{data-planes} expressible as software.  P4 has gained
rapid adoption.  The p4.org consortium~\cite{p4org} was created to
steward the language evolution; p4.org currently includes more than
100 organizations in the areas of networking, cloud systems, network
chip design, and academic institutions.  The P4 specification is open
and public~\cite{p416-spec17}.  Reference implementations for
compilers, simulation and debugging tools are available with a
permissive license at the GitHub P4 repository~\cite{p4lang}.  While
initially P4 was designed for programming network switches, its scope
has been broadened to cover a large variety of packet-processing
systems (e.g., network cards, FPGAs, etc.).
