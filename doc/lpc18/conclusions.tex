\section{Challenges and Future Work}\label{sec:conclusions}
In general, our development experience is mirrored by the lessons described 
in~\cite{minao-hspr18} and ~\cite{bertin-netdev17}.

\paragraph{No multi-/broadcast support}
While XDP is able to redirect single frames it does not have the ability to 
clone and redirect packets similar to \texttt{bpf\_clone\_redirect}. This makes 
development of more sophisticated P4 forwarding programs problematic.

\paragraph{The stack size is too small}
More complex XDP programs are rejected by the verifier despite their safeness. 
This is a particular challenge when attempting to implement network function 
chaining or more advanced pipelined packet processing in a single XDP program.

\paragraph{Generic XDP and TCP}
Our testing framework uses virtual Linux interfaces and generic XDP~\cite{genericxdp}
to verify XDP programs. 
Unfortunately, we are unable to test TCP streams as the protocol is not 
supported by this driver~\cite{xdptcp}.
Any program loaded by generic XDP operates after the creation of
the \texttt{skb} and requires the original packet data. Since TCP clones every 
packet and passes the unmodifiable \texttt{skb} clone,  generic XDP is
bypassed and never receives the datagram.

\paragraph{Using libbpf userspace library}
Creating of compilation of eBPF programs in userspace requires substantial 
effort. Many function calls and variables available in sample programs are not 
available as C library and currently the p4c-xdp project copied these utilities
from kernel code or assembled from various online sources. We plan to switch to
use libbpf to control and manage the eBPF programs and maps.

\paragraph{Persistent eBPF maps in namespaces}
[need to verify] When using eBPF programs in namespaces, maps exported via tc 
do not persist across \texttt{ip netns exec} calls. The consequence is that any 
program has to be run in a single shell command, otherwise the eBPF map becomes 
unusable despite the continued existence of the namespace.
